

%%% Choose a language %%%

\newif\ifEN
\ENtrue   % uncomment this for english
%\ENfalse   % uncomment this for czech

%%% Configuration of the title page %%%

\newif\ifMFF
\MFFtrue % comment this out for the version with a big UK university logo
\def\UKName{Charles University in Prague} %this is only used in UK-logo-version
\def\UKFaculty{Faculty of Mathematics and Physics}

% Thesis type names, as used in several places in the title
%\def\ThesisTypeTitle{\ifEN BACHELOR THESIS \else BAKALÁŘSKÁ PRÁCE \fi}
\def\ThesisTypeTitle{\ifEN MASTER THESIS \else DIPLOMOVÁ PRÁCE \fi}
%\def\ThesisTypeTitle{\ifEN RIGOROUS THESIS \else RIGORÓZNÍ PRÁCE \fi}
%\def\ThesisTypeTitle{\ifEN DOCTORAL THESIS \else DISERTAČNÍ PRÁCE \fi}
%\def\ThesisGenitive{\ifEN bachelor \else bakalářské \fi}
\def\ThesisGenitive{\ifEN master \else diplomové \fi}
%\def\ThesisGenitive{\ifEN rigorous \else rigorózní \fi}
%\def\ThesisGenitive{\ifEN doctoral \else disertační \fi}
%\def\ThesisAccusative{\ifEN bachelor \else bakalářskou \fi}
\def\ThesisAccusative{\ifEN master \else diplomovou \fi}
%\def\ThesisAccusative{\ifEN rigorous \else rigorózní \fi}
%\def\ThesisAccusative{\ifEN doctoral \else disertační \fi}



%%% Fill in your details %%%

% (Note: \xxx is a "ToDo label" which makes the unfilled visible. Remove it.)
\def\ThesisTitle{Integration of the Embree Raycasting Library into a CSG Renderer}
\def\ThesisAuthor{Bc. Sebastian Schimper}
\def\YearSubmitted{2021}

% department assigned to the thesis
\def\Department{Department of Software and Computer Science Education}
% Is it a department (katedra), or an institute (ústav)?
\def\DeptType{Department}

\def\Supervisor{doc. Alexander Wilkie, Dr.}
\def\SupervisorsDepartment{Department of Software and Computer
	Science Education}

% Study programme and specialization
\def\StudyProgramme{Computer Science}
\def\StudyBranch{Computer Graphics and Game Development}

\def\Dedication{%
I owe a significant depth of gratitude to my supervisor, Prof. Alexander Wilkie, for the possibility to contribute to the development of a groundbreaking rendering system and for his help, support, and trust in me. I would also like to express sincere thankfulness towards Alban Fichet. His help and feedback were of crucial importance to my work. Additionally, I would like to thank various members of the Computer Graphics Group in Prague for numerous reviews and discussions, and the "Computer Graphics Stack Exchange" users Peter and Simon F, two strangers on the internet, for offering help when no help was available, thus preventing me from going completely nuts. I am grateful to my brother for helping me resolving various issues concerning Latex. Another person I wish to thank is Solange Petracchi, my contact person at Charles University, for getting me out of trouble multiple times during my years of study and for her moral support. 
Finally, I would like to thank my family, friends, and Monika for their love and support during difficult times.
}

\def\AbstractEN{%
Modern High-Performance Ray Casting toolkits, such as the Intel Embree library, a de facto industry standard, are a cornerstone of the high-performance levels seen in current CPU rendering. The purpose of Embree is an easy integration into professional image synthesis environments to accelerate rendering scenes with complex geometry, usually composed of many primitives. Unfortunately, Embree does not offer support for rendering constructive solid geometry (CSG), solids composed of a manageable amount of primitive solids by using set operations. This is a significant drawback since CSG modeling is an intuitive and powerful option for describing complex geometry.
In this thesis, we describe the integration of Embree into the predictive rendering system ART and propose a method for rendering CSG by combining the traversal of Embree's and Art's internal ray acceleration data structures. The tests we conducted with virtual scenes containing CSG not being constructed from triangle meshes showed that our method is competitive with the original ART renderer and often even faster.
% ABSTRACT IS NOT A COPY OF YOUR THESIS ASSIGNMENT!
}

\def\AbstractCS{%
\xxx{You will need to submit both Czech and English abstract to the SIS, no matter what language you use in the thesis. If writing in English, translate the contents of \texttt{\textbackslash{}AbstractEN} into this field. In case you do not speak czech, your supervisor should be able to help you with the translation.}
}

% 3 to 5 keywords (recommended), each enclosed in curly braces.
% Keywords are useful for indexing and searching for the theses by topic.
\def\Keywords{{Raycasting,} {CSG}, {Embree}}


% If your abstracts are long and do not fit in the infopage, you can make the
% fonts a bit smaller by this setting. (Also, you should try to compress your abstract more.)
% Alternatively, consider increasing the size of the page by uncommenting the
% geometry modification in thesis.tex.
\def\InfoPageFont{}
%\def\InfoPageFont{\small}  %uncomment to decrease font size

\ifEN\relax\else
% If you are writing a czech thesis, you additionally need to fill in the
% english translation of the metadata here!
\def\ThesisTitleEN{\xxx{Thesis title in English}}
\def\DepartmentEN{\xxx{Name of the department in English}}
\def\DeptTypeEN{\xxx{Department}}
\def\SupervisorsDepartmentEN{\xxx{Superdepartment}}
\def\StudyProgrammeEN{\xxx{study programme}}
\def\StudyBranchEN{\xxx{study branch}}
\def\KeywordsEN{%
\xxx{{key} {words}}
}
\fi
